\documentclass{article}

\usepackage{tabularx}
\usepackage{booktabs}

\title{Problem Statement and Goals\\\progname}

\author{\authname}

\date{}

%% Comments

\usepackage{color}

\newif\ifcomments\commentstrue %displays comments
%\newif\ifcomments\commentsfalse %so that comments do not display

\ifcomments
\newcommand{\authornote}[3]{\textcolor{#1}{[#3 ---#2]}}
\newcommand{\todo}[1]{\textcolor{red}{[TODO: #1]}}
\else
\newcommand{\authornote}[3]{}
\newcommand{\todo}[1]{}
\fi

\newcommand{\wss}[1]{\authornote{blue}{SS}{#1}} 
\newcommand{\plt}[1]{\authornote{magenta}{TPLT}{#1}} %For explanation of the template
\newcommand{\an}[1]{\authornote{cyan}{Author}{#1}}

%% Common Parts

\newcommand{\progname}{Student Evaluation App} % PUT YOUR PROGRAM NAME HERE
\newcommand{\authname}{Team 29
\\ Nicholas Fabugais-Inaba
\\ Casra Ghazanfari
\\ Alex Verity
\\ Jung Woo Lee} % AUTHOR NAMES                  

\usepackage{hyperref}
    \hypersetup{colorlinks=true, linkcolor=blue, citecolor=blue, filecolor=blue,
                urlcolor=blue, unicode=false}
    \urlstyle{same}
                                


\begin{document}

\maketitle

\begin{table}[hp]
\caption{Revision History} \label{TblRevisionHistory}
\begin{tabularx}{\textwidth}{llX}
\toprule
\textbf{Date} & \textbf{Developer(s)} & \textbf{Change}\\
\midrule
September 23, 2024 & NFI, JL, CG, AV & Initial Draft\\
Date2 & Name(s) & Description of changes\\
... & ... & ...\\
\bottomrule
\end{tabularx}
\end{table}

\section{Problem Statement}

\subsection{Background}

The McMaster GSA softball league is used every summer by 30-40 teams and
as many as 1,000 unique participants. The league is currently organized through
an old software platform accessible via a web browser, but it is outdated and
does not include features for administrators to maintain the site without
extensive knowledge of computer programming. The GSA league is aware that
paid-for and ad-supported services are available, and features present in
those applications should be explored and added if possible. The GSA league
is a minimal-cost non-profit and would like a personalized platform by which
to operate without committing to paid-for services. Some players find the
current website UI difficult to use and unstable, and would prefer a more
intuitive solution that needs little to no maintenance.

\subsection{Problem}

The platform will be responsible for including all functionalities
of the current solution such as scheduling, division management,
communication between captains, waiver management, rescheduling, score and
league standings management, and other tasks identified by the stakeholders.
Our solution would be an updated form of the existing website's capabilities
with a modernized UI and the additional features of player-specific logins,
real-time standings, and commisioner announcements. Additionally, enhanced
stability is key in replacing the lack of maintainability of the current
website. 

\subsection{Inputs and Outputs}

\subsubsection{Inputs}

\begin{itemize}
    \item Player/Captain/Commissioner login information
    \item Player/Captain/Commissioner contact information
    \item Team information
    \item Game score
    \item Captain/Team availability
    \item Reschedule requests
    \item Commissioner announcements
\end{itemize}

\subsubsection{Outputs}

\begin{itemize}
    \item League standings
    \item League scheduling
    \item Commissioner announcements
\end{itemize}

\subsection{Stakeholders}

\begin{itemize}
    \item The supervisor of the project, Dr. Jake Nease
    \item Commissioners of the league
    \item Captains/Players/Umpires of the softball league
\end{itemize}

\subsection{Environment}

\begin{description}
    \item [Software] Windows, Linux or Mac OS
    \item [Hardware] Computers with access to the internet
\end{description}

\section{Goals}

\begin{description}
    \item [Accomplish everything the existing league website does]
    The current website allows captains to log in and record their matches
    and scores. It allows scheduling and rescheduling, and provides a place
    to see the league rules, parking information and other information. The
    current website often breaks, requiring the current website admin to fix
    issues as they arise. First and foremost we need to recreate the original
    league website functionality.
    \item [User interface should be intuitive to all users.] The current
    interface is unintuiative and awkward to use. Users should understand
    how to log in and how to view their schedule just by looking at their
    homepage. No external information should be required.
    \item [Allow players to make accounts] Currently, only captains have
    accounts in the system. Player accounts should only be able to view the
    contact information of their team captain, captains should only be able
    to view the contact of their players and other captains, and commisioners
    should be able to see everything.
    \item [Matches should be able to be scheduled and rescheduled.] 
    Team captains should be able to give their team's availablity and the
    software will algorithmically schedule the season's matches. If a team
    isn't available for a match after it has been scheduled, captains can
    send a reschedule request with a selection of possible alternative times
    that the opposition team's captain can agree to.
    \item [Commisioners should be able to notify captains with information]
    Commisioner level accounts should be able to easily send out a
    notification to specific users or entire groups of users, such as all
    captains or all players. The information in the notification should be
    customizable by the commisioner.
\end{description}

\section{Stretch Goals}

\begin{description}
    \item [Commisioners should be able to "rain out" matches] After a match
    has been scheduled, commisioner level accounts should be able to force
    a reschedule if the weather makes the game unreasonable to play. This
    will send a notification to the two team captains so they can
    choose a date that works.
    \item [League template saving] A season's teams and players should be
    able to be saved as a template that can be loaded the next season. This
    is useful as many teams remain the same or similar between seasons, and
    it would be convienient for all returning teams to avoid reinviting all
    returning players.
    \item [A mobile application companion] Users would be able to perform
    some actions they can on the website, like viewing schedules and
    standings from their mobile device.
\end{description}

\section{Challenge Level and Extras}

\subsection{Challenge Level}
\begin{description}
    \item[Challenge level:] General
    \item[Rationale:] Does not involve any extensive research and it is an
    improvement of an already existing solution.
\end{description}

\subsection{Extras}
\begin{itemize}
    \item Code walkthroughs
    \item User documentation
\end{itemize}

\newpage{}

\section*{Appendix --- Reflection}

\wss{Not required for CAS 741}

The purpose of reflection questions is to give you a chance to assess your own
learning and that of your group as a whole, and to find ways to improve in the
future. Reflection is an important part of the learning process.  Reflection is
also an essential component of a successful software development process.  

Reflections are most interesting and useful when they're honest, even if the
stories they tell are imperfect. You will be marked based on your depth of
thought and analysis, and not based on the content of the reflections
themselves. Thus, for full marks we encourage you to answer openly and honestly
and to avoid simply writing ``what you think the evaluator wants to hear.''

Please answer the following questions.  Some questions can be answered on the
team level, but where appropriate, each team member should write their own
response:


\begin{enumerate}
    \item What went well while writing this deliverable? 
    \item What pain points did you experience during this deliverable, and how
    did you resolve them?
    \item How did you and your team adjust the scope of your goals to ensure
    they are suitable for a Capstone project (not overly ambitious but also of
    appropriate complexity for a senior design project)?
\end{enumerate}  

\subsection*{Reflection -- Nicholas Fabugais-Inaba}

Majority of the deliverable went well resulting from the information
the team had received in our initial meeting with the supervisor of the
project, Dr. Jake Nease. He was able to detail many of the expectations
he had relating to the information required for the deliverable such as
the inputs, outputs, stakeholders, and goals. Only minimal brainstorming
was required from the team to fill in the rest of the information for the
deliverable. \newline

Certain pain points the team did experience, while writing this deliverable,
related to the problem the project would be addressing. Although some of
the information was gathered from the project description, which was listed
in the potential project document, provided to students, the team
still needed to address other crucial information that may not have
initially come to mind. This was resolved from the collective effort of
the team, brainstorming further resolutions as to why this problem needs
our specific solution. \newline

The team was able to adjust the scope of the goals, based on complexity,
by separating the primary goals of the project with the stretch goals.
Goals such as replicating the current features the exisiting website
possesses, having an easy-to-use interface, and a login system, make sure
the Capstone project isn't too overly ambitious and definitely achievable.
The stretch goals help to add complexity to create a senior design project
as developing a mobile version of the system and other additional features
are extra add-ons that add to the bulk of the work needed to be completed.

\end{document}